\documentclass{article}
\title{The \texttt{cv} class and \texttt{cvsidebar} package}
\author{Erik Weitenberg}

\setlength\parskip{1em}
\setlength\parindent{0pt}
\begin{document}
\maketitle

\section{Introduction}
The \texttt{cv} class allows one to typeset a simple curriculum vitae (cv).
An additional package \texttt{cvsidebar} is provided for a more modern design with a side-bar on the first page.

\section{The cv class}
\subsection{Class options}
The class supports some options, detailed below.
\begin{itemize}
	\item \texttt{headingcolor=}\textit{color} \\
		Change the color of the headings in the document, using the \texttt{xcolor} package.
		You can also use a custom color, provided you define it in the preamble.
		The default is black.
	\item \texttt{selection=}\textit{string} \\
		An arbitrary string, printed at the bottom of each page.
		I use this to identify old printed versions of the cv, since I typically create a custom version of the document for each prospective employer.
		By printing the version on the document, I can later find it back easily if I want to reuse it.
		The default \texttt{all} suppresses the footer.
	\item \texttt{scale=}\textit{float} \\
		A scaling factor, 1 by default, that allows you to increase or decrease the amount of whitespace between lines in the tables.
\end{itemize}

\subsection{New commands}
There are some new commands specific to the creation of a cv.
The following should be used in the preamble.
All of them are optional unless otherwise noted.
\begin{itemize}
	\item \texttt{\textbackslash{}author\{}\textit{name}\texttt{\}} \\
		The same command as in the \texttt{article} class. Mandatory.
	\item \texttt{\textbackslash{}titlebefore\{}\textit{title}\texttt{\}} \\
		A title, like Dr., to be printed before your name.
	\item \texttt{\textbackslash{}titleafter\{}\textit{title}\texttt{\}} \\
		A title, like Msc, to be printed after your name.
	\item \texttt{\textbackslash{}dateofbirth\{}\textit{text}\texttt{\}} \\
	      \texttt{\textbackslash{}address\{}\textit{text}\texttt{\}} \\
	      \texttt{\textbackslash{}phonenumber\{}\textit{text}\texttt{\}} \\
	      \texttt{\textbackslash{}emailaddress\{}\textit{text}\texttt{\}[}\textit{text}\texttt{]} \\
		Several bits of personal data usually present on a cv.
		If you supply them, they will be printed below your name.
		The address is turned into a link in the pdf; if you insert markup code, you should provide the address in plain text in the optional argument to get a working link.
	\item \texttt{\textbackslash{}photofile\{}\textit{filename}\texttt{\}} \\
		If you supply a photo, it will be placed next to your name.
\end{itemize}

The activities you wish to include on your cv can be given using specialized commands, to align them neatly in a table.
You can start such a table using the \texttt{cvtable} environment, which accepts an optional argument to give it a heading. For example,
\begin{verbatim}
\begin{cvtable}[Education]
  ...
\end{cvtable}
\end{verbatim}
Within this environment, you can list your activities and publications using these commands.
\begin{itemize}
	\item \texttt{\textbackslash{}activity*\{}\textit{date}\texttt{\}*\{}\textit{job title}\texttt{\}[}\textit{employer}\texttt{]} \\
		Inserts an activity, like education or employment.
		The first star, if present, causes this row to take less space, which looks nice if you will not include an explanation below it.
		The second star, if present, causes the \textit{job title} not to be bolded, which I find useful if I am not including many explanations.
	\item \texttt{\textbackslash{}explanation\{}\textit{text}\texttt{\}} \\
		Explanatory text, typically inserted after an activity.
	\item \texttt{\textbackslash{}publication\{}\textit{date}\texttt{\}\{}\textit{author}\texttt{\}\{}\textit{title}\texttt{\}\{}\textit{venue}\texttt{\}[}\textit{volume}\texttt{][}\textit{pages}\texttt{]} \\
		Cites a publication. The last two arguments are optional, but you must use none or both.
\end{itemize}

To insert a page break inside a \texttt{cvtable}, use the \texttt{\textbackslash{}pagebreak} command.

There is also a \texttt{cvshorttable} command, which is meant for skills and languages and prints its contents in two columns.
Use \texttt{\textbackslash{}colbreak} inside it to go to the second column.

\section{The cvsidebar package}
\subsection{Package options}
The package supports some options, detailed below.
\begin{itemize}
	\item \texttt{sidebarcolor=}\textit{color} \\
		Change the color of the sidebar, using the \texttt{xcolor} package.
		You can also use a custom color, provided you define it in the preamble.
		The default is \texttt{Gray!30!white}.
	\item \texttt{sidebarwidth=}\textit{length} \\
		Width of the sidebar, 6 cm by default.
\end{itemize}

\subsection{New commands}
There are some new commands specific to the creation of a cv.

The sidebar is created using the sidebar environment, within which you can provide additional text to be printed in the sidebar.
Typically, one would list languages or computer skills here.
\begin{verbatim}
\begin{sidebar}
\begin{cvtable}[Languages]
  \activity{English}*{Native}
  \activity{Spanish}*{Good}
\end{cvtable}
\end{sidebar}
\end{verbatim}
The package provides the additional command \texttt{\textbackslash{}activityFlushRight}, which works exactly like \texttt{\textbackslash{}activity} but flushes the second argument right.
This typically looks nice in the sidebar, especially if you use icons for the skills.

\textbf{The margins of the first page are different to make space for the sidebar.
For this reason, you must end the first page explicitly using the \texttt{\textbackslash{}pagebreak} command.}

\end{document}